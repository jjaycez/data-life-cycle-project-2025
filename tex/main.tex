\documentclass[12pt, letterpaper]{report}

% --- Paquetes y Comandos ---
% --- Paquetes de Idioma y Formato ---
\usepackage[utf8]{inputenc}
\usepackage[spanish, es-tabla]{babel}
\usepackage[T1]{fontenc}
\usepackage{xspace}
\usepackage{geometry}
\geometry{margin=1in}
\usepackage{helvet}
\renewcommand{\familydefault}{\sfdefault}
\usepackage{setspace}
\usepackage{comment}
\onehalfspacing

% --- Paquetes Técnicos ---
\usepackage{amsmath}
\usepackage{graphicx}
\usepackage{booktabs}
\usepackage{hyperref}

% --- Estilos ---
\newcommand{\parsp}{\hspace*{2em}}
\setlength{\parindent}{0pt}
% Setup for syntax highlighted Python code
\usepackage{minted}
\setminted{
  fontsize=\small,
  linenos,
  breaklines
}

% --- DATA SOURCES ---
\newcommand{\openaq}{\texttt{OpenAQ}\xspace}
\newcommand{\datagov}{\texttt{data.gov}\xspace}
\newcommand{\bboxt}{\texttt{bbox}\xspace}

% --- Información del Proyecto ---
\title{\textbf{Impacto de la Contaminación Atmosférica en los Casos de Asma en California}}
\author{\textbf{Grupo Morado}\\
    Julio Agüero Encinas\\
    Fabián Camargo\\
    Miguel Corvo Domosti\\
    Samuel Martín Martínez\\
    Jiaze Zhang}
\date{Enero 2026}

\newif\ifdraft
\drafttrue % Cambia a \draftfalse para compilar el documento completo

\begin{document}
	
	\maketitle
    
    \begin{abstract}
    El crecimiento económico asociado a la industrialización y a la intensificación de la actividad urbana ha
    contribuido a una degradación progresiva de la calidad del aire, con efectos directos sobre la salud pública.
    Entre las consecuencias más relevantes se encuentra el incremento de enfermedades respiratorias y el empeoramiento
    de patologías preexistentes, como el asma, especialmente en entornos con alta densidad logística e industrial.
    
    Este proyecto analiza, tomando California (EE. UU.) como caso de estudio, si los niveles de contaminación
    atmosférica se ajustan a las recomendaciones de las guías globales de la Organización Mundial de la Salud y, en
    paralelo, explora su relación con la incidencia de asma en la población.
    Para ello, se emplearán series históricas de calidad del aire para contaminantes clave (incluyendo materia
    particulada fina, ozono a nivel del suelo y dióxido de nitrógeno, entre otros), junto con datos epidemiológicos de
    asma desagregados por edad, género y localización.
    El trabajo contempla además una evaluación espacial de la cobertura de los puntos de medición, con el fin de
    identificar áreas donde la monitorización resulte insuficiente y proponer mejoras.
    Finalmente, se desarrollarán herramientas de análisis y visualización que faciliten la interpretación de patrones
    temporales y geográficos y apoyen un plan de seguimiento de la salud respiratoria en función de los niveles de contaminación.
    \end{abstract}

	
	\tableofcontents
 
    \ifdraft
    % Use this for draft mode, compiling only the chapters in editing
        \chapter{Introducción}\label{ch:introduccion}


\section{Descripción del Problema}\label{sec:descripcion-del-problema}

En las últimas décadas, el modelo de éxito de las sociedades modernas se ha vinculado frecuentemente a la expansión económica y al crecimiento industrial. Sin embargo, este enfoque genera consecuencias adversas en ocasiones dificilmente apreciables a primera vista: la omisión sistemática de las consecuencias ambientales en favor de la productividad. A medida que las ciudades se expanden y las actividades comerciales se intensifican, el impacto sobre el entorno natural se desplaza a un segundo plano, considerándose a menudo como un efecto secundario inevitable del desarrollo.

Este descuido del entorno tiene usualmente una repercusión directa y alarmante en la salud de la población. La prioridad por sostener una economía de alto rendimiento ha derivado en una degradación de la calidad del aire que respiramos. Al no prestar debidamente atención a las emisiones derivadas de estos sistemas, se han creado condiciones ambientales que no solo facilitan la aparición de nuevas enfermedades respiratorias, sino que agravan seriamente las ya existentes, como es el caso del asma.

En regiones con un alto desarrollo logístico e industrial, como California, el aire se carga de contaminantes que actúan como detonantes de crisis respiratorias.
        \chapter{Fuentes de datos}\label{ch:fuentes-de-datos}
% TODO: decidir si es un título adecuado, he usado uno genérico

En este capítulo se describen las fuentes de datos utilizadas en el análisis del impacto de la contaminación
atmosférica en los casos de asma en California.

Se detallan las características de cada conjunto de datos, su origen, y cómo se integran en el estudio.
En este estudio se han empleado principalmente dos fuentes de datos:
\begin{itemize}
    \item Datos de calidad del aire obtenidos de la iniciativa~\openaq.
    \item
        Datos de prevalencia de asma en la población californiana proporcionados por \datagov, la plataforma de datos
        abiertos del gobierno de Estados Unidos.
\end{itemize}

Para la extracción y filtrado de los datos, se han realizado una serie de tareas que se describirán en detalle en las
secciones siguientes.

Todo este proceso ha sido recogido en un cuaderno de Jupyter, que se puede localizar en el repositorio del proyecto,
en la ruta \texttt{code/ETL.ipynb}.

\bigskip

A lo largo de las secciones, se facilitarán algunas capturas de pantalla de fragmentos del cuaderno para ilustrar
ciertas partes del proceso.

\section{Datos de calidad del aire}\label{sec:datos-de-calidad-del-aire}

Los datos de calidad del aire se han obtenido de la iniciativa~\openaq, que proporciona acceso abierto a
mediciones de contaminantes atmosféricos a nivel global.

\smallskip

Entre otros contaminantes, las entidades colaboradoras exponen datos sobre niveles de dióxido de nitrógeno ($NO_{2}$),
ozono ($O_{3}$) y material particulado fino ($PM_{2.5}$), que son relevantes para el estudio del asma.

\smallskip

Además de ofrecer un visor web, \openaq cuenta con una API que permite la descarga programática de datos históricos
y en tiempo real.
En la documentación oficial de la API se detallan los \textit{endpoints} disponibles, los parámetros de consulta y los
formatos de respuesta.

\smallskip

En concreto, es de particular relevancia el hecho de que la API tiene un límite de peticiones para distintas franjas
temporales.
En el caso de California, ha sido de relevancia considerar los siguientes límites, cuyas razones revelaremos más adelante:

\begin{itemize}
    \item 60 peticiones por minuto.
    \item 2000 peticiones por hora.
\end{itemize}

Estas restricciones han sido el principal cuello de botella a la hora de descargar grandes volúmenes de datos
históricos.

\subsection{Geografía de California}\label{subsec:geografia-california}

Los datos expuestos por el gobierno de Estados Unidos están desglosados por condados, que son las subdivisiones
administrativas de primer nivel dentro de los estados.

\bigskip

Esto es de relevancia, puesto que las estaciones de medición de calidad del aire exponen su ubicación geográfica en
coordenadas de latitud y longitud (EPSG:4326), por lo que es necesario asignar cada estación a su condado
correspondiente.

Para este cometido, se han realizado dos tareas principales:

\begin{itemize}
    \item
        \textbf{Obtener los límites geográficos de los condados de California.}
        
        Para ello, se han empleado los datos de \textit{shapefiles} proporcionados por el
        \textit{U.S. Census Bureau}\footnote{\url{https://www2.census.gov/geo/tiger/TIGER2025/COUNTY/tl_2025_us_county.zip}}.
        % TODO: buscar cómo integrarlo en el sistema de referencias bibliográficas
        Estos archivos contienen la geometría de los condados, que se ha utilizado para determinar si una estación
        de medición se encuentra dentro de un condado específico.
        
    \item
        \textbf{Filtrar las estaciones de medición mediante un \bboxt}

        (\textit{bounding box}, i.e., un rectángulo delimitador) que englobe toda California, puesto que la API de
        \openaq permite filtrar por este criterio.
\end{itemize}

\begin{comment}
    - Describir el formato de los shapefiles
    - Indicar que California se corresponde con el estado con código FIPS 06
    - Indicar que se ha empleado `shapely` para determinar si un punto está dentro de un polígono
    - Mostrar una distribución de estaciones por condado → mejor lo movemos más adelante
\end{comment}

A continuación se muestra el código empleado para descargar los shapefiles de los condados:

\begin{minted}{python}

import requests
from pathlib import Path
# download county shapefiles from US Census Bureau
shapes_url = 'https://www2.census.gov/geo/tiger/TIGER2025/COUNTY/tl_2025_us_county.zip'
geo_dir = Path("data/geographical")
geo_dir.mkdir(parents=True, exist_ok=True)
shapes_path = geo_dir / "tl_2025_us_county.zip"
if not shapes_path.exists():
    print(f"Downloading county shapefiles from {shapes_url}...")
    r = requests.get(shapes_url)
    with open(shapes_path, 'wb') as f:
        f.write(r.content)
    print(f"Downloaded to {shapes_path}")
else:
    print(f"Shapefiles already exist at {shapes_path}")
   
\end{minted}

El archivo comprimido descargado se encuentra en la ruta \path{data/geographical/tl_2025_us_county.zip},
cuya versión descomprimida se halla en la sub-carpeta \path{california_counties}; en caso de que el lector desee
explorar los archivos.

\bigskip

En la Figura~\ref{fig:us_counties_qgis} se muestra una captura de pantalla de los condados de Estados Unidos
visualizados en QGIS, un software de sistemas de información geográfica (\textit{GIS}).

Los condados de California están resaltados en rosa.

\bigskip

\begin{figure}[h]
    \centering
    \includegraphics[width=0.8\linewidth]{figures/us_counties_qgis}
    \caption{Visualización de condados en QGIS}
    \label{fig:us_counties_qgis}
\end{figure}

Son de particular interés las siguientes columnas:

\begin{itemize}
    \item \texttt{STATEFP}: código FIPS del estado (California es 06).
    
    \item \texttt{NAME}: nombre del condado, valor que empleamos para relacionarlo con los datos de asma.
    
    \item
        \texttt{geometry}: geometría del condado en formato poligonal, es decir, una cadena de texto en formato WKT
        (\textit{Well-Known Text}).
    
        Se facilita un ejemplo ilustrativo en la Figura~\ref{fig:wkt-ejemplo}.
   
\end{itemize}

\begin{figure}[ht]
    \centering
    \begin{minted}[fontsize=\small,frame=single]{text}
        POLYGON ((-122.373121 37.883884, -122.371142 37.884364, ... ))
    \end{minted}
    \caption{Ejemplo: geometría en WKT (ilustrativo)}
    \label{fig:wkt-ejemplo}
\end{figure}

Para procesar estos datos geográficos, se ha empleado la librería
\mintinline{python}{geopandas}, que extiende las funcionalidades de \mintinline{python}{pandas} para manejar
datos espaciales.

En concreto, nos permite usar el método \mintinline{python}{Polygon.contains()} para determinar si las coordenadas
de una estación de medición están dentro de la geometría de un condado.

\bigskip

Una vez que tenemos definido el flujo a seguir para clasificar las estaciones por condado, procedemos a exponer la
cuestión de decidir cómo filtrar las estaciones de interés a través de la API de \openaq.

Como ya se ha mencionado, la API permite filtrar las estaciones mediante un \textit{bounding box}, que se define
por las coordenadas de sus esquinas suroeste y noreste.

Para California, simplemente se ha usado QGIS para obtener unas coordenadas aproximadas que engloban todo el estado.
En la Figura~\ref{fig:california_bbox_qgis} se muestra una captura de pantalla de QGIS donde se procede a copiar las
coordenadas.

\begin{figure}[H]
    \centering
    \includegraphics[width=1\linewidth]{figures/california_bbox_qgis}
    \caption{Obtención del \bboxt de California en QGIS}
    \label{fig:california_bbox_qgis}
\end{figure}

Se puede consultar el código empleado en el Apéndice~\ref{ch:apendice-a:-codigo-para-la-fase-etl}, en el bloque de
Código~\ref{lst:openaq_extraction}.

Posteriormente, estos datos se han procesado para asignar cada estación a su condado correspondiente, como se ha
descrito anteriormente.

Se proporciona una tabla con la distribución de estaciones por condado en California (se ha mostrado solo
los 10 primeros condados ordenados por recuento) en la Tabla~\ref{tab:county_counts}.

\begin{table}[ht]
    \centering
    \caption{Recuento por condado}
    \label{tab:county_counts}
    \begin{tabular}{lr}
        \toprule
        Condado & Recuento \\
        \midrule
        Los Angeles      & 403 \\
        Alameda          & 190 \\
        Mono             &  94 \\
        Contra Costa     &  87 \\
        Sacramento       &  83 \\
        San Francisco    &  66 \\
        Monterey         &  50 \\
        Sonoma           &  48 \\
        San Diego        &  46 \\
        San Luis Obispo  &  44 \\
        \bottomrule
    \end{tabular}
\end{table}


\subsection{Descarga de datos históricos}\label{subsec:descarga-datos-historicos}

\begin{comment}
    - Explicar la cabecera `X-Rate-Limit-Remaining` y cómo se ha empleado para gestionar las peticiones
    - Explicar el tiempo estimado para descargar los datos de toda California
    - Explicar que el código se tuvo que modificar varias veces, pero no se perdió el progreso gracias a los archivos
      intermedios, por eso la barra de progreso muestra un tiempo menor del real
    - Explicar que son 5333 archivos JSON, por lo que se ha optado comprimirlos en un ZIP para facilitar su manipulación
    - Mostrar como ejemplo la estructura de una respuesta en formato JSON
\end{comment}

Ahora que tenemos los datos de las estaciones de interés, el siguiente paso es realizar una petición a la API
de \openaq para descargar los datos históricos de calidad del aire.

La API devuelve información sobre los sensores de cada estación, que miden distintos contaminantes atmosféricos.

A partir de esto, se ha diseñado un bucle que itera sobre los 5333 sensores identificados en California, empleando el
\textit{endpoint} \texttt{/sensors} de la API.

Dado que la API impone límites en el número de peticiones por minuto y por hora, se ha implementado una lógica
para monitorizar las cabeceras de respuesta \texttt{X-Rate-Limit-Remaining} y pausar las peticiones cuando sea necesario.

Este proceso ha tenido una naturaleza iterativa en cuanto al código, puesto que han surgido varios inconvenientes
durante la descarga masiva de datos.

Sin embargo, puesto que se han guardado los datos descargados en archivos intermedios, no se ha perdido el progreso
realizado hasta el momento.

Es por esta razón que la barra de progreso muestra un tiempo estimado menor del real.

Teniendo en cuenta los límites impuestos por la API, se estima que la descarga completa de los datos históricos
de calidad del aire en California lleva entre 2 y 3 horas.

Se adjunta una captura de pantalla en la Figura~\ref{fig:openaq_download_progress} que ilustra el progreso de la descarga.

\begin{figure}[H]
    \centering
    \includegraphics[width=0.8\linewidth]{figures/openaq_download_progress}
    \caption{Celda del cuaderno ilustrando las peticiones}
    \label{fig:openaq_download_progress}
\end{figure}

Puesto que se han guardado los datos descargados en archivos JSON individuales, se ha optado por comprimirlos en un archivo ZIP
para facilitar su manipulación posterior.

Además, puesto que no se van a modificar, facilita su almacenamiento y distribución.

El bucle se puede consultar en el cuaderno de Jupyter.
No lo incluimos aquí por su extensión.

En la Figura ~\ref{fig:openaq_sample_json} se muestra un ejemplo de la estructura de una respuesta en formato JSON, cuyo
contenido ha sido truncado para mayor claridad.

\begin{figure}[H]
    \centering
    \begin{minted}[fontsize=\small,frame=single]{json}
[
    {
        "value": 9.0,
        "flagInfo": {
            "hasFlags": false
        },
        "parameter": {
            "id": 2,
            "name": "pm25",
            "units": "\u00b5g/m\u00b3",
            "displayName": null
        },
        "period": {
            "label": "raw",
            "interval": "01:00:00",
            "datetimeFrom": {
                "utc": "2016-03-06T19:00:00Z",
                "local": "2016-03-06T11:00:00-08:00"
            },
            "datetimeTo": {
                "utc": "2016-03-06T20:00:00Z",
                "local": "2016-03-06T12:00:00-08:00"
            }
        },
        ...
    }
]
    \end{minted}
    \caption{Ejemplo de respuesta JSON de la API de \openaq}
    \label{fig:openaq_sample_json}
\end{figure}


\subsection{Procesamiento de los datos descargados}\label{subsec:procesamiento-datos-descargados}

\begin{comment}
    - Explicar que se ha creado una base de datos SQLite para facilitar el análisis posterior
    - Mostrar un esquema de la base de datos
    - REMINDER: los datos no han sido curados, solo se han insertado tal cual
    - Explicar, si procede, las decisiones de diseño tomadas (tipos de datos, índices, etc.)
\end{comment}

Con la idea de integrar los datos de calidad del aire con los datos de asma, se ha creado una base de datos SQLite
para facilitar el análisis posterior.

Se ha diseñado un esquema de base de datos que incluye tablas para las estaciones de medición, los sensores, las mediciones
realizadas … entre otros.

En la Figura~\ref{fig:database_schema} se muestra un esquema generado por DBeaver de la base de datos.

\begin{figure}[H]
    \centering
    \includegraphics[width=0.7\linewidth]{../data/openaq_data/sensors/aq_data_cal}
    \caption{Esquema de la BD en la Fase I}
    \label{fig:database_schema}
\end{figure}

Más adelante, la base de datos se ampliará para incluir los datos de asma, permitiendo realizar consultas
complejas que relacionen ambos conjuntos de datos.

\newpage

\section{Datos de prevalencia de asma en California}\label{sec:datos-asma}

\begin{comment}
    - Explicar la estructura de los datos obtenidos de data.gov
    - TODO: procesar los datos en la base de datos SQLite
    - Mostrar un esquema actualizado de la base de datos
    - Explicar las decisiones de diseño tomadas (tipos de datos, índices, etc.)
\end{comment}


        
\chapter{Plan de Preservación}\label{ch:plan-de-preservacion}

El presente Plan de Preservación tiene como objetivo garantizar la integridad, disponibilidad y reutilización a largo plazo de los activos digitales generados en el proyecto \textit{Impacto de la Contaminación Atmosférica en los Casos de Asma en California}. Dado que el estudio maneja series temporales críticas y datos de salud pública, se ha diseñado una estrategia que cubre desde el almacenamiento físico hasta la gestión de derechos de acceso, alineándose con los principios FAIR (\textit{Findable, Accessible, Interoperable, Reusable}).

\section{Almacenamiento}\label{sec:almacenamiento}

La estrategia de almacenamiento se ha diseñado teniendo en cuenta el volumen de los datos geoespaciales y la necesidad de diferenciar claramente entre los datos de uso frecuente y los archivos destinados a la preservación estática.

\subsection{Almacenamiento físico}\label{subsec:almacenamiento-fisico}

La infraestructura física local constituye el primer nivel de nuestra jerarquía de datos. Debido a la carga computacional que requieren las operaciones espaciales en \textbf{ArcGIS Pro} —especialmente el cruce de capas vectoriales de códigos postales (ZCTA) con las series temporales de contaminantes—, hemos priorizado el rendimiento en los equipos de trabajo.\\

Para el \textbf{almacenamiento activo}, utilizamos unidades de estado sólido (\textbf{SSD}) en las estaciones de trabajo locales. Esto nos permite evitar cuellos de botella durante la renderización de mapas y la ejecución de los scripts de limpieza en R. Adicionalmente, mantenemos un nivel de \textbf{almacenamiento en frío} (\textit{cold storage}) mediante discos duros externos mecánicos (HDD) de alta capacidad. En estos dispositivos se custodian las copias originales de los \textit{raw data} descargados de la EPA y \textit{catalog.data.gov}, liberando espacio en los discos de trabajo principales y sirviendo como repositorio local de los datos fuente inalterados.

\section{Gestión de los datos}\label{sec:gestion-de-los-datos}

El presente estudio se adhiere a los principios FAIR (Findable, Accessible, Interoperable, Reusable) para garantizar la transparencia y la reproducibilidad de la investigación. La gestión de la información no se limita a la recolección, sino que abarca la totalidad del ciclo de vida de los datos, estableciendo procedimientos claros para su manejo desde la obtención inicial hasta su preservación a largo plazo.

\subsection{Limpieza y curación}\label{subsec:limpieza-y-curacion}

La transformación de los datos brutos en información analizable se ha llevado a cabo mediante un flujo de trabajo reproducible ejecutado en el entorno estadístico \textbf{R}. En lugar de correcciones manuales, que son propensas a errores y difíciles de auditar, hemos desarrollado scripts automatizados utilizando la librería \texttt{tidyverse} para estandarizar los formatos de fecha y unificar las unidades de medida de los contaminantes ($NO_2$, $O_3$) a partes por millón (ppm).\\

Para el tratamiento de inconsistencias, los scripts aplican reglas de validación lógica que identifican registros duplicados y valores fuera de rango. Específicamente, la detección de \textit{outliers} en las lecturas de los sensores se realiza mediante el cálculo del rango intercuartílico (IQR), filtrando automáticamente aquellas mediciones que exceden el umbral de $1.5 \times IQR$ por considerarse errores instrumentales. Respecto a los valores nulos (NaNs) en las series temporales, cuando los huecos de información son inferiores a 4 horas, se imputan mediante algoritmos de interpolación lineal disponibles en el paquete \texttt{zoo}, garantizando la continuidad de la serie sin introducir sesgos significativos.\\

Por último, la integridad espacial de los datos se verifica utilizando \textbf{ArcGIS Pro}. Antes de realizar el cruce espacial entre las estaciones de monitoreo y los códigos postales (ZCTA), empleamos las herramientas de topología de ArcGIS para asegurar que las geometrías de los \textit{shapefiles} no presentan superposiciones ni huecos que pudieran alterar la asignación de los niveles de exposición.

\subsection{Metadatos}\label{subsec:metadatos}

Para facilitar la recuperación e interpretación de los datos por terceros, se ha adoptado el esquema de metadatos estandarizado \textbf{Dublin Core}. La elección de este estándar internacional permite describir los recursos digitales mediante un conjunto de elementos esenciales —tales como Título, Creador, Cobertura Espacial, Fecha y Formato—, garantizando que el contexto de recolección y las características técnicas del \textit{dataset} sean comprensibles sin ambigüedades para la comunidad científica.\\

Como parte fundamental de esta documentación, se adjunta un diccionario de datos o \textit{codebook} detallado que define explícitamente cada variable, especifica las unidades de medida empleadas (ej. ppm o $\mu g/m^3$) y desglosa el significado de cualquier codificación utilizada en las tablas. Adicionalmente, para asegurar el orden, todos los archivos siguen una convención de nombrado sistemática (ej. \texttt{YYYYMMDD\_Nombre\_v01}) que facilita la identificación rápida de versiones.

\subsection{Acceso y reutilización}\label{subsec:acceso-y-reutilizacion}

Una vez finalizado el estudio, los datos procesados se depositarán en el repositorio de acceso abierto \textbf{Zenodo}, gestionado por el CERN. Esta plataforma asignará automáticamente un Identificador de Objeto Digital (DOI) único al conjunto de datos, asegurando que sea localizable y citable de manera permanente en futuras investigaciones.\\

Los datos se distribuirán bajo la licencia \textbf{Creative Commons Atribución 4.0 Internacional (CC-BY 4.0)}. Esta licencia abierta fomenta la reutilización, permitiendo a terceros compartir y adaptar el material para cualquier propósito, incluso comercial, con la única condición de reconocer la autoría original. En cuanto a las consideraciones éticas, al trabajar con datos de prevalencia agregados a nivel de condado y no con historias clínicas individuales, se garantiza la privacidad de los pacientes sin necesidad de aplicar técnicas adicionales de anonimización sobre los resultados finales.

\section{Seguridad}\label{sec:seguridad}

La seguridad de la información en este proyecto se ha abordado desde una perspectiva integral que contempla tanto la integridad de los ficheros como la privacidad de los datos sensibles, aspectos críticos cuando se trabaja con información relacionada con la salud pública. Aunque los datos de prevalencia de asma utilizados son agregados y no contienen información personal identificable (PII) de pacientes individuales, hemos aplicado protocolos estrictos para evitar cualquier riesgo de reidentificación o manipulación malintencionada de los resultados.\\

En primer lugar, garantizamos la integridad de los datos mediante el uso de funciones de resumen o \textit{hashing}. Cada vez que se descarga un conjunto de datos original de las fuentes gubernamentales o se genera un \textit{dataset} consolidado tras la limpieza, se calcula su huella digital. Esto nos permite verificar periódicamente que los archivos almacenados en nuestros discos locales no han sufrido alteraciones silenciosas, conocidas como \textit{bit rot}, ni modificaciones accidentales durante su manipulación. Si la suma de verificación actual no coincide con la registrada originalmente, el sistema nos alerta para restaurar una copia limpia desde el sistema de copias de seguridad.\\

Por otro lado, el control de acceso a los datos en fase de desarrollo se gestiona mediante permisos estrictos en el repositorio de código. Hemos establecido una política de roles donde únicamente los miembros del equipo de investigación tienen permisos de escritura y modificación sobre los \textit{scripts} de análisis y los datos brutos. Cualquier cambio en el código que procesa los datos de contaminación o salud debe pasar por una revisión por pares antes de ser fusionado con la rama principal del proyecto. Esta trazabilidad completa nos asegura que ningún dato ha sido alterado de forma arbitraria para forzar una correlación estadística inexistente, manteniendo así la ética y la validez científica del estudio.\\

Finalmente, para el tránsito de información entre los equipos locales y los sistemas de almacenamiento remoto, utilizamos exclusivamente protocolos cifrados. Tanto la sincronización con la nube institucional como las operaciones de confirmación de cambios en el repositorio de código se realizan a través de canales seguros HTTPS y SSH, protegiendo la propiedad intelectual del proyecto y los datos de posibles interceptaciones en redes no seguras.

\section{Estrategia de preservación a largo plazo}\label{sec:estrategia-de-preservacion-a-largo-plazo}

La preservación digital a largo plazo es un desafío que va más allá del simple almacenamiento de archivos; implica asegurar que la información permanezca legible, comprensible y ejecutable a medida que la tecnología evoluciona. Nuestra estrategia se fundamenta en la migración proactiva de formatos y la documentación exhaustiva, con el objetivo de que este estudio sobre el asma en California pueda ser reproducido con exactitud dentro de diez o veinte años, independientemente del software que exista en ese momento.

Una de las decisiones más importantes que hemos tomado para garantizar esta perdurabilidad es la renuncia al uso de formatos propietarios para el archivo definitivo. Aunque durante la fase activa del proyecto utilizamos formatos nativos de Excel o archivos de proyecto de ArcGIS Pro por su eficiencia operativa, somos conscientes de que estos formatos podrían quedar obsoletos o requerir licencias de software costosas en el futuro. Por ello, nuestra política de preservación dicta que todo activo digital debe tener una versión equivalente en un estándar abierto:
\begin{itemize}
    \item \textbf{Datos tabulares:} Se convierten a \textbf{CSV} con codificación UTF-8, asegurando su legibilidad por cualquier editor de texto básico.
    \item \textbf{Datos geográficos:} La cartografía vectorial se exporta a \textbf{GeoJSON}, un formato basado en texto y ampliamente soportado.
    \item \textbf{Documentación:} La memoria y los manuales se archivan en \textbf{PDF/A}, el estándar ISO diseñado específicamente para el archivo a largo plazo que incrusta todas las fuentes y elementos visuales necesarios.
\end{itemize}

\subsection{Backup}\label{subsec:backup}

Para mitigar el riesgo de pérdida catastrófica de datos, ya sea por fallos de hardware, errores humanos o ataques de software malicioso, hemos implementado una política de copias de seguridad rigurosa basada en el estándar de la industria conocido como la regla \textbf{3-2-1}. Esta metodología asegura que no exista un único punto de fallo que pueda comprometer la totalidad del proyecto.

El protocolo establecido dicta que debemos mantener en todo momento al menos \textbf{tres} copias completas de todos los datos, almacenadas en \textbf{dos} soportes de diferente naturaleza, con \textbf{una} copia ubicada fuera de sitio (\textit{off-site}):

\begin{enumerate}
    \item \textbf{Copia de Trabajo:} Reside en las unidades SSD locales para el procesamiento diario.
    \item \textbf{Copia de Seguridad Local (Air Gap):} Se realiza semanalmente en los discos duros externos (HDD). La característica clave de esta copia es su aislamiento: los discos se mantienen \textbf{desconectados física y lógicamente} de la red y de la corriente eléctrica cuando no se está realizando la copia. Esta técnica, conocida como "brecha de aire" (\textit{air gap}), proporciona nuestra defensa más robusta contra el \textit{ransomware}, ya que es físicamente imposible que un software malicioso encripte una unidad que no está conectada al sistema infectado.
    \item \textbf{Copia Remota:} Se almacena en la nube institucional, asegurando la recuperación ante desastres físicos en el lugar de trabajo, como incendios o robos.
\end{enumerate}

\subsection{Almacenamiento en la nube}\label{subsec:almacenamiento-en-la-nube}

El almacenamiento en la nube juega un doble rol fundamental en nuestro proyecto: facilita la colaboración distribuida durante la fase activa y actúa como repositorio final para la preservación estática.

Durante el desarrollo, utilizamos \textbf{Google Drive/OneDrive} (vinculados a cuentas institucionales) y \textbf{GitHub} para la sincronización inmediata de documentos y código entre los miembros del equipo. Sin embargo, para la preservación a largo plazo, dependemos de \textbf{Zenodo}. La elección de este repositorio del CERN no es trivial: a diferencia de las nubes comerciales que requieren pagos recurrentes o mantenimiento de cuentas activas, Zenodo garantiza la preservación de los archivos científicos durante al menos 20 años, asegurando que la evidencia generada en este estudio permanezca accesible a la comunidad científica de manera perpetua.
    \else
	% --- INTRODUCCIÓN ---
        \chapter{Introducción}\label{ch:introduccion}


\section{Descripción del Problema}\label{sec:descripcion-del-problema}

En las últimas décadas, el modelo de éxito de las sociedades modernas se ha vinculado frecuentemente a la expansión económica y al crecimiento industrial. Sin embargo, este enfoque genera consecuencias adversas en ocasiones dificilmente apreciables a primera vista: la omisión sistemática de las consecuencias ambientales en favor de la productividad. A medida que las ciudades se expanden y las actividades comerciales se intensifican, el impacto sobre el entorno natural se desplaza a un segundo plano, considerándose a menudo como un efecto secundario inevitable del desarrollo.

Este descuido del entorno tiene usualmente una repercusión directa y alarmante en la salud de la población. La prioridad por sostener una economía de alto rendimiento ha derivado en una degradación de la calidad del aire que respiramos. Al no prestar debidamente atención a las emisiones derivadas de estos sistemas, se han creado condiciones ambientales que no solo facilitan la aparición de nuevas enfermedades respiratorias, sino que agravan seriamente las ya existentes, como es el caso del asma.

En regiones con un alto desarrollo logístico e industrial, como California, el aire se carga de contaminantes que actúan como detonantes de crisis respiratorias.
        
        % --- GESTIÓN DEL PROYECTO ---
        \chapter{Gestión del proyecto}\label{ch:gestion-del-proyecto}


\section{Planificación y Paquetes de Trabajo}\label{sec:planificacion-y-paquetes-de-trabajo}

Siguiendo la metodología de división lógica de actividades, el proyecto se organiza en 5 paquetes de trabajo (WP)[cite: 150, 156]:
\begin{itemize}
    \item \textbf{WP1:} Concepción y desarrollo del proyecto[cite: 158].
    \item \textbf{WP2:} Gestión y redacción del informe[cite: 161, 174].
    \item \textbf{WP3:} Obtención y procesado de datos clínicos y ambientales[cite: 163].
    \item \textbf{WP4:} Análisis estadístico y conclusiones[cite: 164, 179].
    \item \textbf{WP5:} Promoción y presentación[cite: 166].
\end{itemize}

\section{Plan de gestión de datos (DMP)}\label{sec:plan-de-gestion-de-datos-(dmp)}

Se aplican los principios FAIR (Findable, Accessible, Interoperable, Reusable) para asegurar una gestión efectiva y transparente de la información[cite: 187, 191]. El cumplimiento de HIPAA asegura que no se utilice material sensible que identifique pacientes[cite: 212].
        
        % --- FUENTES DE DATOS ---
        \chapter{Fuentes de datos}\label{ch:fuentes-de-datos}
% TODO: decidir si es un título adecuado, he usado uno genérico

En este capítulo se describen las fuentes de datos utilizadas en el análisis del impacto de la contaminación
atmosférica en los casos de asma en California.

Se detallan las características de cada conjunto de datos, su origen, y cómo se integran en el estudio.
En este estudio se han empleado principalmente dos fuentes de datos:
\begin{itemize}
    \item Datos de calidad del aire obtenidos de la iniciativa~\openaq.
    \item
        Datos de prevalencia de asma en la población californiana proporcionados por \datagov, la plataforma de datos
        abiertos del gobierno de Estados Unidos.
\end{itemize}

Para la extracción y filtrado de los datos, se han realizado una serie de tareas que se describirán en detalle en las
secciones siguientes.

Todo este proceso ha sido recogido en un cuaderno de Jupyter, que se puede localizar en el repositorio del proyecto,
en la ruta \texttt{code/ETL.ipynb}.

\bigskip

A lo largo de las secciones, se facilitarán algunas capturas de pantalla de fragmentos del cuaderno para ilustrar
ciertas partes del proceso.

\section{Datos de calidad del aire}\label{sec:datos-de-calidad-del-aire}

Los datos de calidad del aire se han obtenido de la iniciativa~\openaq, que proporciona acceso abierto a
mediciones de contaminantes atmosféricos a nivel global.

\smallskip

Entre otros contaminantes, las entidades colaboradoras exponen datos sobre niveles de dióxido de nitrógeno ($NO_{2}$),
ozono ($O_{3}$) y material particulado fino ($PM_{2.5}$), que son relevantes para el estudio del asma.

\smallskip

Además de ofrecer un visor web, \openaq cuenta con una API que permite la descarga programática de datos históricos
y en tiempo real.
En la documentación oficial de la API se detallan los \textit{endpoints} disponibles, los parámetros de consulta y los
formatos de respuesta.

\smallskip

En concreto, es de particular relevancia el hecho de que la API tiene un límite de peticiones para distintas franjas
temporales.
En el caso de California, ha sido de relevancia considerar los siguientes límites, cuyas razones revelaremos más adelante:

\begin{itemize}
    \item 60 peticiones por minuto.
    \item 2000 peticiones por hora.
\end{itemize}

Estas restricciones han sido el principal cuello de botella a la hora de descargar grandes volúmenes de datos
históricos.

\subsection{Geografía de California}\label{subsec:geografia-california}

Los datos expuestos por el gobierno de Estados Unidos están desglosados por condados, que son las subdivisiones
administrativas de primer nivel dentro de los estados.

\bigskip

Esto es de relevancia, puesto que las estaciones de medición de calidad del aire exponen su ubicación geográfica en
coordenadas de latitud y longitud (EPSG:4326), por lo que es necesario asignar cada estación a su condado
correspondiente.

Para este cometido, se han realizado dos tareas principales:

\begin{itemize}
    \item
        \textbf{Obtener los límites geográficos de los condados de California.}
        
        Para ello, se han empleado los datos de \textit{shapefiles} proporcionados por el
        \textit{U.S. Census Bureau}\footnote{\url{https://www2.census.gov/geo/tiger/TIGER2025/COUNTY/tl_2025_us_county.zip}}.
        % TODO: buscar cómo integrarlo en el sistema de referencias bibliográficas
        Estos archivos contienen la geometría de los condados, que se ha utilizado para determinar si una estación
        de medición se encuentra dentro de un condado específico.
        
    \item
        \textbf{Filtrar las estaciones de medición mediante un \bboxt}

        (\textit{bounding box}, i.e., un rectángulo delimitador) que englobe toda California, puesto que la API de
        \openaq permite filtrar por este criterio.
\end{itemize}

\begin{comment}
    - Describir el formato de los shapefiles
    - Indicar que California se corresponde con el estado con código FIPS 06
    - Indicar que se ha empleado `shapely` para determinar si un punto está dentro de un polígono
    - Mostrar una distribución de estaciones por condado → mejor lo movemos más adelante
\end{comment}

A continuación se muestra el código empleado para descargar los shapefiles de los condados:

\begin{minted}{python}

import requests
from pathlib import Path
# download county shapefiles from US Census Bureau
shapes_url = 'https://www2.census.gov/geo/tiger/TIGER2025/COUNTY/tl_2025_us_county.zip'
geo_dir = Path("data/geographical")
geo_dir.mkdir(parents=True, exist_ok=True)
shapes_path = geo_dir / "tl_2025_us_county.zip"
if not shapes_path.exists():
    print(f"Downloading county shapefiles from {shapes_url}...")
    r = requests.get(shapes_url)
    with open(shapes_path, 'wb') as f:
        f.write(r.content)
    print(f"Downloaded to {shapes_path}")
else:
    print(f"Shapefiles already exist at {shapes_path}")
   
\end{minted}

El archivo comprimido descargado se encuentra en la ruta \path{data/geographical/tl_2025_us_county.zip},
cuya versión descomprimida se halla en la sub-carpeta \path{california_counties}; en caso de que el lector desee
explorar los archivos.

\bigskip

En la Figura~\ref{fig:us_counties_qgis} se muestra una captura de pantalla de los condados de Estados Unidos
visualizados en QGIS, un software de sistemas de información geográfica (\textit{GIS}).

Los condados de California están resaltados en rosa.

\bigskip

\begin{figure}[h]
    \centering
    \includegraphics[width=0.8\linewidth]{figures/us_counties_qgis}
    \caption{Visualización de condados en QGIS}
    \label{fig:us_counties_qgis}
\end{figure}

Son de particular interés las siguientes columnas:

\begin{itemize}
    \item \texttt{STATEFP}: código FIPS del estado (California es 06).
    
    \item \texttt{NAME}: nombre del condado, valor que empleamos para relacionarlo con los datos de asma.
    
    \item
        \texttt{geometry}: geometría del condado en formato poligonal, es decir, una cadena de texto en formato WKT
        (\textit{Well-Known Text}).
    
        Se facilita un ejemplo ilustrativo en la Figura~\ref{fig:wkt-ejemplo}.
   
\end{itemize}

\begin{figure}[ht]
    \centering
    \begin{minted}[fontsize=\small,frame=single]{text}
        POLYGON ((-122.373121 37.883884, -122.371142 37.884364, ... ))
    \end{minted}
    \caption{Ejemplo: geometría en WKT (ilustrativo)}
    \label{fig:wkt-ejemplo}
\end{figure}

Para procesar estos datos geográficos, se ha empleado la librería
\mintinline{python}{geopandas}, que extiende las funcionalidades de \mintinline{python}{pandas} para manejar
datos espaciales.

En concreto, nos permite usar el método \mintinline{python}{Polygon.contains()} para determinar si las coordenadas
de una estación de medición están dentro de la geometría de un condado.

\bigskip

Una vez que tenemos definido el flujo a seguir para clasificar las estaciones por condado, procedemos a exponer la
cuestión de decidir cómo filtrar las estaciones de interés a través de la API de \openaq.

Como ya se ha mencionado, la API permite filtrar las estaciones mediante un \textit{bounding box}, que se define
por las coordenadas de sus esquinas suroeste y noreste.

Para California, simplemente se ha usado QGIS para obtener unas coordenadas aproximadas que engloban todo el estado.
En la Figura~\ref{fig:california_bbox_qgis} se muestra una captura de pantalla de QGIS donde se procede a copiar las
coordenadas.

\begin{figure}[H]
    \centering
    \includegraphics[width=1\linewidth]{figures/california_bbox_qgis}
    \caption{Obtención del \bboxt de California en QGIS}
    \label{fig:california_bbox_qgis}
\end{figure}

Se puede consultar el código empleado en el Apéndice~\ref{ch:apendice-a:-codigo-para-la-fase-etl}, en el bloque de
Código~\ref{lst:openaq_extraction}.

Posteriormente, estos datos se han procesado para asignar cada estación a su condado correspondiente, como se ha
descrito anteriormente.

Se proporciona una tabla con la distribución de estaciones por condado en California (se ha mostrado solo
los 10 primeros condados ordenados por recuento) en la Tabla~\ref{tab:county_counts}.

\begin{table}[ht]
    \centering
    \caption{Recuento por condado}
    \label{tab:county_counts}
    \begin{tabular}{lr}
        \toprule
        Condado & Recuento \\
        \midrule
        Los Angeles      & 403 \\
        Alameda          & 190 \\
        Mono             &  94 \\
        Contra Costa     &  87 \\
        Sacramento       &  83 \\
        San Francisco    &  66 \\
        Monterey         &  50 \\
        Sonoma           &  48 \\
        San Diego        &  46 \\
        San Luis Obispo  &  44 \\
        \bottomrule
    \end{tabular}
\end{table}


\subsection{Descarga de datos históricos}\label{subsec:descarga-datos-historicos}

\begin{comment}
    - Explicar la cabecera `X-Rate-Limit-Remaining` y cómo se ha empleado para gestionar las peticiones
    - Explicar el tiempo estimado para descargar los datos de toda California
    - Explicar que el código se tuvo que modificar varias veces, pero no se perdió el progreso gracias a los archivos
      intermedios, por eso la barra de progreso muestra un tiempo menor del real
    - Explicar que son 5333 archivos JSON, por lo que se ha optado comprimirlos en un ZIP para facilitar su manipulación
    - Mostrar como ejemplo la estructura de una respuesta en formato JSON
\end{comment}

Ahora que tenemos los datos de las estaciones de interés, el siguiente paso es realizar una petición a la API
de \openaq para descargar los datos históricos de calidad del aire.

La API devuelve información sobre los sensores de cada estación, que miden distintos contaminantes atmosféricos.

A partir de esto, se ha diseñado un bucle que itera sobre los 5333 sensores identificados en California, empleando el
\textit{endpoint} \texttt{/sensors} de la API.

Dado que la API impone límites en el número de peticiones por minuto y por hora, se ha implementado una lógica
para monitorizar las cabeceras de respuesta \texttt{X-Rate-Limit-Remaining} y pausar las peticiones cuando sea necesario.

Este proceso ha tenido una naturaleza iterativa en cuanto al código, puesto que han surgido varios inconvenientes
durante la descarga masiva de datos.

Sin embargo, puesto que se han guardado los datos descargados en archivos intermedios, no se ha perdido el progreso
realizado hasta el momento.

Es por esta razón que la barra de progreso muestra un tiempo estimado menor del real.

Teniendo en cuenta los límites impuestos por la API, se estima que la descarga completa de los datos históricos
de calidad del aire en California lleva entre 2 y 3 horas.

Se adjunta una captura de pantalla en la Figura~\ref{fig:openaq_download_progress} que ilustra el progreso de la descarga.

\begin{figure}[H]
    \centering
    \includegraphics[width=0.8\linewidth]{figures/openaq_download_progress}
    \caption{Celda del cuaderno ilustrando las peticiones}
    \label{fig:openaq_download_progress}
\end{figure}

Puesto que se han guardado los datos descargados en archivos JSON individuales, se ha optado por comprimirlos en un archivo ZIP
para facilitar su manipulación posterior.

Además, puesto que no se van a modificar, facilita su almacenamiento y distribución.

El bucle se puede consultar en el cuaderno de Jupyter.
No lo incluimos aquí por su extensión.

En la Figura ~\ref{fig:openaq_sample_json} se muestra un ejemplo de la estructura de una respuesta en formato JSON, cuyo
contenido ha sido truncado para mayor claridad.

\begin{figure}[H]
    \centering
    \begin{minted}[fontsize=\small,frame=single]{json}
[
    {
        "value": 9.0,
        "flagInfo": {
            "hasFlags": false
        },
        "parameter": {
            "id": 2,
            "name": "pm25",
            "units": "\u00b5g/m\u00b3",
            "displayName": null
        },
        "period": {
            "label": "raw",
            "interval": "01:00:00",
            "datetimeFrom": {
                "utc": "2016-03-06T19:00:00Z",
                "local": "2016-03-06T11:00:00-08:00"
            },
            "datetimeTo": {
                "utc": "2016-03-06T20:00:00Z",
                "local": "2016-03-06T12:00:00-08:00"
            }
        },
        ...
    }
]
    \end{minted}
    \caption{Ejemplo de respuesta JSON de la API de \openaq}
    \label{fig:openaq_sample_json}
\end{figure}


\subsection{Procesamiento de los datos descargados}\label{subsec:procesamiento-datos-descargados}

\begin{comment}
    - Explicar que se ha creado una base de datos SQLite para facilitar el análisis posterior
    - Mostrar un esquema de la base de datos
    - REMINDER: los datos no han sido curados, solo se han insertado tal cual
    - Explicar, si procede, las decisiones de diseño tomadas (tipos de datos, índices, etc.)
\end{comment}

Con la idea de integrar los datos de calidad del aire con los datos de asma, se ha creado una base de datos SQLite
para facilitar el análisis posterior.

Se ha diseñado un esquema de base de datos que incluye tablas para las estaciones de medición, los sensores, las mediciones
realizadas … entre otros.

En la Figura~\ref{fig:database_schema} se muestra un esquema generado por DBeaver de la base de datos.

\begin{figure}[H]
    \centering
    \includegraphics[width=0.7\linewidth]{../data/openaq_data/sensors/aq_data_cal}
    \caption{Esquema de la BD en la Fase I}
    \label{fig:database_schema}
\end{figure}

Más adelante, la base de datos se ampliará para incluir los datos de asma, permitiendo realizar consultas
complejas que relacionen ambos conjuntos de datos.

\newpage

\section{Datos de prevalencia de asma en California}\label{sec:datos-asma}

\begin{comment}
    - Explicar la estructura de los datos obtenidos de data.gov
    - TODO: procesar los datos en la base de datos SQLite
    - Mostrar un esquema actualizado de la base de datos
    - Explicar las decisiones de diseño tomadas (tipos de datos, índices, etc.)
\end{comment}


        
        % --- PROCESAMIENTO DE DATOS ---
        \chapter{Procesamiento de datos}\label{ch:procesamiento-de-datos}


\section{Limpieza y curación}\label{sec:limpieza-y-curacion}

Se utiliza el lenguaje R para la limpieza de valores nulos (NaNs) e inconsistencias[cite: 284, 285]. La integración geográfica se realiza mediante sistemas SIG como ArcGIS Pro[cite: 287].

\section{Preservación}\label{sec:preservacion}

Los datasets se preservarán en repositorios abiertos como Zenodo, asignando un DOI para facilitar su citación[cite: 327, 328].
        
        % ---ANÁLISIS DE DATOS ---
        \chapter{Análisis de datos}\label{ch:analisis-de-datos}


\section{Análisis gráfico}\label{sec:analisis-grafico}

Se observarán mapas de calor de contaminantes en relación con núcleos urbanos de California[cite: 347, 365].

\section{Análisis estadístico}\label{sec:analisis-estadistico}

Se emplearán árboles de regresión y ajustes lineales para determinar la capacidad explicativa del modelo ambiental sobre los casos de asma[cite: 425, 428].
        
        % --- CONCLUSIONES ---
        \chapter{Conclusiones}\label{ch:conclusiones}

Se discutirán los resultados obtenidos, reconociendo limitaciones como el número de estaciones de medida o variables socioeconómicas no consideradas[cite: 550, 560].
Se propondrán recomendaciones para futuras estrategias de salud pública[cite: 577].
    \fi
    
	% --- APÉNDICES ---
	\appendix
	\chapter{Código para la fase ETL}\label{ch:apendice-a:-codigo-para-la-fase-etl}
En este apéndice se presenta el código utilizado para la fase de Extracción, Transformación y Carga (ETL)
de los datos en nuestro proyecto.

Las variables no definidas en los fragmentos de código se pueden consultar en la siguiente lista:
\begin{itemize}
    \item \texttt{API\_KEY}: Clave de API para acceder a la API de \openaq.
    \item \texttt{selected\_stations}: DataFrame de pandas que contiene las estaciones seleccionadas para la descarga de datos históricos.
\end{itemize}

\begin{listing}
    \begin{minted}{python}
# define bounding box for California
x_range = (-125, -113)
y_range = (32, 42.5)

import requests

BASE = "https://api.openaq.org/v3"
headers = {"X-API-Key": API_KEY}

all_results = []
page = 1
limit = 1000   # OpenAQ v3 maximum

while True:
    r = requests.get(
        f"{BASE}/locations",
        headers=headers,
        params={
            "bbox": f"{x_range[0]},{y_range[0]},{x_range[1]},{y_range[1]}",
            "limit": limit,
            "page": page,
            "country": "US",   # optional but helps validation
        },
        timeout=60,
    )
    r.raise_for_status()
    data = r.json()

    results = data.get("results", [])
    if not results:
        break

    all_results.extend(results)
    print(f"page {page}: {len(results)}")

    page += 1

print("TOTAL returned:", len(all_results))
print("first name:", all_results[0].get("name"))
    \end{minted}
    \caption{Petición a la API de \openaq dentro de un \bboxt}
    \label{lst:openaq_extraction}
\end{listing}

	
    % --- BIBLIOGRAFÍA ---
    \bibliographystyle{plainurl}
    \bibliography{bibliography}
\end{document}