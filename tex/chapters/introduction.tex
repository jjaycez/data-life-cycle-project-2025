\chapter{Introducción}\label{ch:introduccion}


\section{Descripción del Problema}\label{sec:descripcion-del-problema}

En las últimas décadas, el modelo de éxito de las sociedades modernas se ha vinculado frecuentemente a la expansión económica y al crecimiento industrial. Sin embargo, este enfoque genera consecuencias adversas en ocasiones dificilmente apreciables a primera vista: la omisión sistemática de las consecuencias ambientales en favor de la productividad. A medida que las ciudades se expanden y las actividades comerciales se intensifican, el impacto sobre el entorno natural se desplaza a un segundo plano, considerándose a menudo como un efecto secundario inevitable del desarrollo.

Este descuido del entorno tiene usualmente una repercusión directa y alarmante en la salud de la población. La prioridad por sostener una economía de alto rendimiento ha derivado en una degradación de la calidad del aire que respiramos. Al no prestar debidamente atención a las emisiones derivadas de estos sistemas, se han creado condiciones ambientales que no solo facilitan la aparición de nuevas enfermedades respiratorias, sino que agravan seriamente las ya existentes, como es el caso del asma.

En regiones con un alto desarrollo logístico e industrial, como California, el aire se carga de contaminantes que actúan como detonantes de crisis respiratorias.