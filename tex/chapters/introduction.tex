\chapter{Aspectos generales del proyecto}\label{ch:introduccion}

\section{Problemática y objetivos}\label{sec:problematica-y-objetivos}

\subsection{Problemática}\label{subsec:problematica}
En las últimas décadas, el modelo de éxito de las sociedades modernas se ha vinculado frecuentemente a la expansión
económica fundamentada en la industrialización.
Tradicionalmente, esta estrategia no ha prestado atención a los efectos colaterales pertinentes a la salud pública y al medio ambiente.

A medida que las ciudades se expanden y las actividades comerciales se intensifican, el impacto sobre el entorno natural
se desplaza a un segundo plano, considerándose a menudo como un efecto secundario inevitable del desarrollo.

Este descuido del entorno tiene usualmente una repercusión directa y alarmante en la salud de la población.
La prioridad por sostener una economía de alto rendimiento ha derivado en una degradación de la calidad del aire que respiramos.
Al no prestar debidamente atención a las emisiones derivadas de estos sistemas, se han creado condiciones ambientales que
no solo causan la aparición de nuevas enfermedades respiratorias, sino que agravan seriamente las ya existentes,
como es el caso del asma.

En regiones con un alto desarrollo logístico e industrial, los contaminantes que actúan como detonantes
de crisis respiratorias.

\subsection{Objetivos}\label{subsec:objetivos}

El objetivo principal del proyecto es analizar si en California (EEUU) la calidad del aire está dentro de los niveles
recomendados por la OMS ~\cite{who2021_global_aqg}, que establecen unas recomendaciones sobre los niveles
de materia particulada, ozono, óxidos de nitrógeno, azufre y monóxido de carbono.
Esto se manifiesta en límites horarios, octo-horarios, diarios y anuales.

Para este proyecto, son de especial interés los siguientes contaminantes:

\begin{itemize}
    \item \textbf{Dióxido de nitrógeno} ($\notwo$)
    
    Ciertos estudios lo relacionan con un aumento del riesgo en el desarrollo de asma durante la infancia, como~\cite{Naidoo2019NO2asthma}.)
    
    También se ha observado lo mismo en adultos, aunque con mayor heterogeneidad en los resultados~\cite{Khreis2017TRAPchildhoodAsthma}.
    
    Si bien en este último estudio se reconoce que las conclusiones son limitadas, puesto que muchos estudios se apoyan
    en asma definida por cuestionarios y no en diagnósticos clínicos, entre otros factores.
    
    \item \textbf{Materia particulada fina} ($\pmtwofive$~y~$\pmten$)
    
    Su exposición aumenta el riesgo de desarrollo de asma en la infancia.
    
    En el caso de~$\pmtwofive$~es más homogéneo y exacerbado a partir de los 6 años continuados de exposición.
    
    En el caso de~$\pmten$~se ha observado un aumento del riesgo en todos los grupos de edad~\cite{Khreis2017TRAPchildhoodAsthma}.
    
    \item \textbf{Ozono} ($\ozone$) \textbf{a nivel del suelo}
    
    Debido a la reducción en capacidad pulmonar de los pacientes asmáticos, tras la exposición a este contaminante, se
    ha observado un aumento en las crisis asmáticas y en las hospitalizaciones relacionadas con el asma~\cite{epa2025_ozone_asthma_health_effects}.
\end{itemize}


\subsubsection{Objetivos específicos}\label{subsubsec:objetivos-especificos}
Para alcanzar el objetivo principal, se plantean los siguientes objetivos específicos:

\begin{itemize}
    \item Recopilar y procesar datos históricos de calidad del aire en California, centrándose en los contaminantes
    mencionados anteriormente.
    
    \item Evaluar la cobertura espacial de los puntos de medición existentes y proponer la instalación de nuevos puntos
    en áreas con deficiencias de monitorización.
    
    \item Analizar la relación entre los niveles de contaminación atmosférica y la incidencia de asma en la población
    de California, desglosada por edad, género y ubicación geográfica.
    
    \item Desarrollar herramientas de análisis y visualización que faciliten la interpretación de los resultados y
    permitan identificar patrones temporales y geográficos.
    
    \item Proponer un plan para monitorizar la salud respiratoria de la población en función de los niveles de
    contaminación atmosférica.
\end{itemize}

\section{Resultados esperados}\label{sec:resultados-esperados}

Los resultados esperados se pueden resumir en los siguientes puntos:

\begin{enumerate}
    \item \textbf{Informe detallado sobre la calidad del aire en California}
    
    Elaboración de un informe exhaustivo que detalle los niveles de los contaminantes analizados, comparándolos con las
    recomendaciones de la OMS y destacando las áreas con mayores deficiencias.
    
    Además, relacionarlo con la incidencia de asma en la población, identificando posibles correlaciones y tendencias.
    Evaluar su impacto en la salud pública.
    
    \item \textbf{Análisis de la cobertura de monitorización}
    
    Evaluación de la distribución geográfica de los puntos de medición de calidad del aire en California, identificando
    áreas con deficiencias en la monitorización y proponiendo la instalación de nuevos puntos en dichas zonas.
    
    \item \textbf{Modelo predictivo de ingresos hospitalarios por asma}
    
    Desarrollo de un modelo predictivo que relacione los niveles de contaminación atmosférica con la incidencia de
    asma en la población, con el fin de anticipar posibles crisis sanitarias y planificar recursos médicos.
    
    \item \textbf{Base de datos integrada}
    
    Creación de una base de datos integrada que combine datos de calidad del aire y datos epidemiológicos, facilitando
    futuros análisis y estudios.
    Junto con flujo de trabajo reproducible para la actualización periódica de los datos.
    
    \item \textbf{Base científica para políticas públicas en California}
    
    Proporcionar una base científica sólida que respalde la formulación de políticas públicas destinadas a mejorar la
    calidad del aire y la salud respiratoria de la población en California.
    
\end{enumerate}

Este último punto es especialmente relevante, ya que los resultados del proyecto pueden influir en la toma de decisiones
políticas, promoviendo medidas que reduzcan la contaminación atmosférica y mejoren la calidad de vida de los habitantes de California.

\section{Requisitos generales y técnicos}\label{sec:requisitos}
Para llevar a cabo este proyecto, se requieren los siguientes recursos y condiciones:

\begin{enumerate}
    \item \textbf{Datos}
    
    Se requiere acceso a bases de datos históricas de calidad del aire en California, que incluyan mediciones de los
    contaminantes de interés en cada condado de California.
    Los datos deben estar preferiblemente tomados horariamente, puesto que a partir de ello se pueden computar
    varias métricas por las que se rige la normativa de la OMS (diarias, octo-horarias, anuales, etc.).
    
    \bigskip
    
    Similarmente, se requieren datos de prevalencia de asma en California por condado.
    Es de interés que los datos incluyan agregación por edad, puesto que varios estudios enfocan sus conclusiones en
    desarrollo de asma en la infancia.
    Además, es deseable que los datos incluyan un intervalo de confianza, puesto que la prevalencia de asma se suele
    estimar a partir de encuestas, y no es factible conducir un censo exhaustivo.
    
    \item \textbf{Herramientas y software}
    
    Se utilizarán herramientas de análisis de datos y visualización, como Python con bibliotecas especializadas
    (pandas, geopandas, etc.).
    Se emplearán cuadernos de Jupyter para documentar el proceso ETL (Extracción, Transformación y Carga) y los análisis realizados.
    Para la elaboración de gráficos y mapas, se emplearán herramientas como Power BI o QGIS.
    Los datos se transformarán a una base de datos SQLite para facilitar su consulta y análisis, con la posibilidad
    de extender a otros sistemas de gestión de bases de datos si el volumen de datos lo requiere.
    
    \item \textbf{Recursos humanos}
    
    Se requieren los siguientes perfiles profesionales para desarrollar el proyecto:
    
    \begin{itemize}
        \item \textbf{Científico de datos}
        
        Responsable de la recopilación, procesamiento y análisis de los datos, así como del desarrollo de modelos predictivos.
        
        \item \textbf{Especialista en salud pública}
        
        Proporcionará conocimientos sobre la epidemiología del asma y su relación con la contaminación atmosférica.
        
        \item \textbf{Visualizador de datos}
        
        Encargado de diseñar y crear visualizaciones efectivas que faciliten la interpretación de los resultados.
        
        \item \textbf{Técnico GIS}
        
        Responsable del análisis espacial de los datos y de la evaluación de la cobertura de monitorización.
        
        \item \textbf{Gestor de proyectos}
        
        Encargado de coordinar las actividades del equipo, gestionar los plazos y asegurar el cumplimiento de los objetivos.
    \end{itemize}
    
    \item \textbf{Recursos materiales}
    
    Se requiere acceso a equipos informáticos con capacidad suficiente para manejar grandes volúmenes de datos y
    realizar análisis complejos.
    Para concretar los requisitos de hardware, se evaluará el volumen de datos a procesar y la complejidad de los análisis
    previstos conjuntamente con el científico de datos.
    Para garantizar un entorno de trabajo colaborativo, se utilizarán plataformas de gestión de proyectos y almacenamiento en la nube.
    
    \item \textbf{Organización cronológica}
    
    El proyecto se desarrollará en varias fases, cada una con sus propios hitos y entregables que se deberán discutir
    detalladamente con el gestor de proyectos y el equipo involucrado.
    Esto irá acompañado de la elaboración de un presupuesto detallado para evitar demoras y sobrecostes.
\end{enumerate}