\chapter{Impacto la calidad del aire sobre la calidad de vida}\label{ch:descripcion-del-proyecto}



\section{Objetivos del proyecto}\label{sec:objetivos-del-proyecto}

El objetivo principal del proyecto será instaurar una monitorización de los contaminantes a lo largo del Estado de California, EEUU, y con ello aumentar el control estricto del impacto negativo de estos en la salud de la población, de acuerdo a los estándares propios de la OMS. Así mismo, se busca aumentar la concienciación de la sociedad hacia las consecuencias de la industrialización en la salud.
\begin{enumerate}
    \item Obtener datos de niveles de $NO_{2}$, $O_{3}$ y $PM_{2.5}$.
    \item Analizar la correlación con ingresos hospitalarios.
    \item Identificar áreas que requieren medidas de mitigación urgentes.
\end{enumerate}

\section{Marco teórico}\label{sec:marco-teorico}

\subsection{Calidad del aire y contaminantes}\label{subsec:calidad-del-aire-y-contaminantes}

Se monitorean gases como el dióxido de nitrógeno ($NO_{2}$) y ozono ($O_{3}$), los cuales son tóxicos y causan problemas como asma y bronquitis[cite: 57, 72].