\chapter{Gestión del proyecto}\label{ch:gestion-del-proyecto}


\section{Planificación y Paquetes de Trabajo}\label{sec:planificacion-y-paquetes-de-trabajo}

Siguiendo la metodología de división lógica de actividades, el proyecto se organiza en 5 paquetes de trabajo (WP)[cite: 150, 156]:
\begin{itemize}
    \item \textbf{WP1:} Concepción y desarrollo del proyecto[cite: 158].
    \item \textbf{WP2:} Gestión y redacción del informe[cite: 161, 174].
    \item \textbf{WP3:} Obtención y procesado de datos clínicos y ambientales[cite: 163].
    \item \textbf{WP4:} Análisis estadístico y conclusiones[cite: 164, 179].
    \item \textbf{WP5:} Promoción y presentación[cite: 166].
\end{itemize}

\section{Plan de gestión de datos (DMP)}\label{sec:plan-de-gestion-de-datos-(dmp)}

Se aplican los principios FAIR (Findable, Accessible, Interoperable, Reusable) para asegurar una gestión efectiva y transparente de la información[cite: 187, 191]. El cumplimiento de HIPAA asegura que no se utilice material sensible que identifique pacientes[cite: 212].