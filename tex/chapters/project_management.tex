\chapter{Estructura de Desglose del Trabajo (WBS)}

Este capítulo describe la \textit{Work Breakdown Structure} (WBS) del proyecto
\textit{Impacto de la calidad del aire en la prevalencia de asma en California}.
La WBS organiza el trabajo en un conjunto de fases secuenciales y coherentes con
el cronograma del proyecto, reflejando el ciclo completo de los datos: desde la
planificación inicial hasta la evaluación final y la propuesta de medidas.

La estructura adoptada permite asegurar trazabilidad entre objetivos, datos,
análisis y resultados, así como facilitar el control del avance y la evaluación
del rendimiento del proyecto.

\section{Definición y planificación del proyecto}

La fase de planificación establece el marco conceptual, metodológico y organizativo
del estudio. En esta etapa se define la problemática de investigación, se concretan
los objetivos generales y específicos, y se delimitan el ámbito geográfico,
temporal y poblacional del análisis.

Asimismo, se diseña el proceso completo de recopilación, curación y análisis de
datos históricos, garantizando coherencia metodológica y reproducibilidad. En este
contexto se definen los indicadores clave de rendimiento (KPIs), los hitos del
proyecto y el cronograma de ejecución, así como los recursos técnicos y humanos
necesarios.

De forma transversal, se establecen mecanismos de control y seguimiento que permiten
gestionar riesgos, documentar decisiones y asegurar la continuidad del proyecto a
lo largo de todas sus fases.

\section{Recopilación y curación de datos}

Esta fase comprende la identificación y adquisición de las fuentes de datos
relevantes, incluyendo información sobre prevalencia de asma, contaminantes
atmosféricos y variables complementarias de carácter demográfico o contextual.
Se evalúa la cobertura, granularidad y calidad inicial de cada fuente, y se
desarrollan scripts automatizados para la recopilación sistemática de los datos.

Los datos brutos se almacenan con control de versiones y preservación de la
integridad, constituyendo una referencia inmutable para el resto del proyecto.
Posteriormente, se lleva a cabo un proceso de curación que incluye el análisis de
la estructura de los datos, la selección del formato de almacenamiento, la
definición de un esquema lógico y la aplicación de reglas de limpieza y
normalización.

El resultado de esta fase es un conjunto de datos curados, documentados mediante
metadatos y preparados para su uso analítico y modelado, junto con un pipeline
automatizado que garantiza la reproducibilidad del proceso.

\section{Análisis de datos}

En la fase de análisis se estudia la relación entre la calidad del aire y la
prevalencia de asma mediante técnicas estadísticas y exploratorias. Se analizan
correlaciones entre contaminantes y variables de salud, así como tendencias
temporales que permitan identificar patrones relevantes y posibles cambios a lo
largo del tiempo.

Los resultados obtenidos se someten a una evaluación preliminar para verificar su
coherencia interna y su consistencia con la literatura existente. A partir de este
análisis se generan visualizaciones y reportes que sintetizan los hallazgos de
forma clara y reproducible. Todo el proceso analítico se automatiza para facilitar
su ejecución bajo distintos escenarios y configuraciones.

\section{Modelo predictivo}

Sobre la base de los datos curados y los resultados del análisis exploratorio, se
desarrolla un modelo predictivo orientado a estimar indicadores de asma a partir de
variables relacionadas con la calidad del aire. Se construyen modelos base que
sirven como referencia y modelos específicos que incorporan información adicional
y relaciones más complejas.

Los modelos se evalúan mediante métricas previamente definidas y se analizan desde
el punto de vista de su interpretabilidad, con el fin de identificar los factores
más influyentes en las predicciones. Los modelos finales y sus artefactos asociados
se almacenan y documentan para su reutilización y validación futura.

\section{Evaluación de la cobertura de sensores}

Esta fase analiza la adecuación de la red de estaciones de medición de contaminantes
para soportar el análisis y el modelado realizados. Se evalúa la cobertura espacial
y temporal de los sensores, identificando puntos críticos y deficiencias específicas
para cada contaminante considerado.

Los resultados de esta evaluación permiten contextualizar las conclusiones del
estudio y fundamentar recomendaciones sobre mejoras en la infraestructura de
medición o sobre la interpretación de los resultados en áreas con cobertura
limitada.

\section{Publicación de resultados y cierre del proyecto}

La fase final integra y difunde los resultados del proyecto. Se recopilan y discuten
los hallazgos derivados del análisis de datos, el modelado predictivo y la
evaluación de cobertura, y se redacta el documento final que presenta el estudio de
forma estructurada.

Finalmente, se publican los conjuntos de datos y el código utilizado en un
repositorio accesible, y se elabora un informe final que evalúa el rendimiento del
proyecto, resume las conclusiones principales y propone posibles líneas de actuación
o investigación futura.
