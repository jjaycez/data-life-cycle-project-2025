\chapter{Estructura de Desglose del Trabajo (WBS)}

Este capítulo describe la \textit{Work Breakdown Structure} (WBS) del proyecto \textit{Impacto de la calidad del aire en la prevalencia de asma en California}.
La WBS organiza el trabajo en fases y paquetes jerárquicos coherentes con el cronograma (Gantt) y con el flujo lógico del ciclo de vida de los datos: planificación, adquisición, curación, análisis, modelado, evaluación de cobertura, publicación y cierre.

Cada paquete de trabajo se define en términos de objetivo, actividades principales y entregables, de modo que el avance del proyecto sea medible y verificable.

\section{Fase 1. Definición y planificación del proyecto}

La primera fase establece las bases conceptuales, metodológicas y organizativas del proyecto. Su objetivo es reducir la incertidumbre inicial y asegurar que el análisis posterior sea coherente, reproducible y alineado con los objetivos planteados.

\subsection{1.1 Descripción de la problemática y definición de objetivos}

Este paquete de trabajo se centra en formular el problema de investigación y concretar los objetivos generales y específicos del estudio.
Se delimitan las preguntas analíticas relacionadas con la asociación entre contaminantes atmosféricos y la prevalencia de asma, así como la población de interés y el ámbito geográfico y temporal del análisis.

El entregable principal es un documento de alcance que recoge las hipótesis de trabajo, los supuestos iniciales y los criterios de éxito del proyecto.

\subsection{1.2 Definición del proceso de recopilación, curación y análisis de datos}

En este paquete se diseña el flujo completo de tratamiento de los datos, desde la recopilación de fuentes primarias hasta la generación de resultados analíticos y modelos predictivos.
Se definen las etapas del pipeline, las dependencias entre ellas y los mecanismos de control de calidad y trazabilidad.

El resultado es una especificación formal del proceso de datos que sirve como referencia para las fases posteriores.

\subsection{1.3 Definición del modelo predictivo}

Este subcomponente establece el enfoque de modelado que se utilizará para estimar la relación entre la calidad del aire y los indicadores de asma.
Incluye la definición de la variable objetivo, la selección de modelos base y modelos específicos, así como los criterios de evaluación y validación.

El entregable es un diseño metodológico del modelo predictivo, alineado con los objetivos del estudio y las limitaciones de los datos disponibles.

\subsection{1.4 Definición del proceso de evaluación de la cobertura de sensores}

Este paquete describe cómo se evaluará la cobertura espacial y temporal de las estaciones de medición de contaminantes.
Se definen métricas de cobertura, criterios para la identificación de puntos críticos y el impacto potencial de las deficiencias de medición en el análisis y el modelado.

El resultado es un marco metodológico para analizar la adecuación de la infraestructura de sensores existente.

\subsection{1.5 Definición de KPIs, hitos y cronograma}

Aquí se establecen los indicadores clave de rendimiento (KPIs) del proyecto, los hitos principales y el cronograma de ejecución.
Estos elementos permiten monitorizar el progreso y evaluar el grado de cumplimiento de los objetivos en cada fase.

\subsection{1.6 Evaluación de recursos necesarios}

Este subpaquete identifica los recursos técnicos y humanos requeridos, incluyendo herramientas de software, capacidad de almacenamiento y cómputo, así como esfuerzos de desarrollo y análisis.

\subsection{1.7 Control y seguimiento del proyecto}

De forma transversal a toda la fase de planificación, se definen mecanismos de seguimiento, gestión de riesgos y control de cambios.
Este componente garantiza la coherencia del proyecto a lo largo del tiempo y facilita la toma de decisiones informadas.

\section{Fase 2. Recopilación de datos}

La segunda fase aborda la identificación, adquisición y almacenamiento inicial de los datos necesarios para el análisis.

\subsection{2.1 Definición de fuentes de datos}

Se identifican y documentan las fuentes de datos relevantes, incluyendo registros de prevalencia de asma, mediciones de contaminantes y variables demográficas complementarias.
Se analizan su cobertura, granularidad y posibles limitaciones.

\subsection{2.2 Generación de scripts para la recopilación automática}

Este paquete se centra en el desarrollo de scripts automatizados para la descarga de datos, garantizando reproducibilidad y consistencia en la adquisición.

\subsection{2.3 Almacenamiento y control de versiones de datos brutos}

Los datos recopilados se almacenan en su forma original, manteniendo control de versiones e integridad, con el fin de preservar una referencia inmutable para las fases posteriores.

\subsection{2.4 Análisis exploratorio inicial}

Se realiza un análisis exploratorio preliminar para identificar problemas evidentes de calidad, cobertura o consistencia que puedan afectar al proceso de curación.

\section{Fase 3. Curación de datos}

Esta fase transforma los datos brutos en conjuntos de datos estructurados, consistentes y listos para el análisis.

\subsection{3.1 Análisis de la estructura de los datos}

Se examina la estructura interna de los datos, verificando tipos, claves, duplicados y coherencia entre fuentes.

\subsection{3.2 Evaluación del formato de almacenamiento}

Se selecciona el formato de almacenamiento más adecuado para los datos curados, considerando eficiencia, escalabilidad y compatibilidad con las herramientas analíticas.

\subsection{3.3 Elaboración del esquema lógico}

Se define el esquema lógico que organiza los datos curados y sus relaciones, sirviendo como base para el análisis y el modelado.

\subsection{3.4 Inserción de datos en el formato definitivo}

Los datos se cargan en el formato seleccionado, aplicando validaciones para asegurar la integridad del proceso.

\subsection{3.5 Tratamiento de valores vacíos y extremos}

Se aplican reglas de limpieza y normalización para gestionar valores faltantes y observaciones extremas.

\subsection{3.6 Almacenamiento de datos curados y metadatos}

Se documentan los datos curados mediante metadatos y descripciones que facilitan su reutilización y comprensión.

\subsection{3.7 Automatización del proceso de curación}

Se integran todas las etapas de curación en un pipeline automatizado y reproducible.

\section{Fase 4. Análisis de datos}

En esta fase se estudian las relaciones entre contaminantes y asma mediante análisis estadísticos y visualizaciones.

\subsection{4.1 Análisis de correlación}

Se evalúa la relación entre niveles de contaminantes y prevalencia de asma en un año base.

\subsection{4.2 Análisis de tendencias temporales}

Se analizan las tendencias a lo largo del tiempo para identificar patrones y cambios relevantes.

\subsection{4.3 Evaluación preliminar de resultados}

Se realiza una validación inicial de los resultados obtenidos, contrastándolos con la literatura y los supuestos del estudio.

\subsection{4.4 Generación de visualizaciones e informes}

Se elaboran visualizaciones y reportes que sintetizan los resultados del análisis.

\subsection{4.5 Automatización del análisis}

El proceso analítico se automatiza para permitir su ejecución repetida bajo distintos escenarios.

\section{Fase 5. Modelo predictivo}

Esta fase desarrolla y evalúa modelos predictivos basados en los datos analizados.

\subsection{5.1 Modelos base}

Se construyen modelos de referencia que establecen un punto de comparación mínimo.

\subsection{5.2 Modelos específicos}

Se implementan modelos más complejos que incorporan múltiples variables explicativas.

\subsection{5.3 Evaluación de modelos}

Se comparan los modelos específicos con los modelos base utilizando métricas definidas.

\subsection{5.4 Interpretabilidad}

Se analizan los factores que más influyen en las predicciones del modelo.

\subsection{5.5 Publicación y almacenamiento de modelos}

Los modelos finales se almacenan y documentan para su reutilización.

\section{Fase 6. Evaluación de cobertura de sensores}

Se analiza la adecuación de la red de sensores para soportar el análisis realizado.

\subsection{6.1 Análisis de cobertura actual}

Se evalúa la cobertura espacial y temporal de las estaciones de medición.

\subsection{6.2 Identificación de puntos críticos}

Se detectan áreas con cobertura insuficiente.

\subsection{6.3 Identificación de deficiencias por contaminante}

Se analizan las carencias específicas para cada contaminante considerado.

\section{Fase 7. Publicación de resultados}

Esta fase consolida y difunde los resultados del proyecto.

\subsection{7.1 Recopilación y discusión de resultados}

Se integran los resultados del análisis, modelado y evaluación de cobertura.

\subsection{7.2 Redacción del artículo}

Se redacta el documento científico o técnico que presenta el estudio.

\subsection{7.3 Publicación de datos y código}

Se publican los datasets y el código en un repositorio accesible.

\section{Fase 8. Evaluación final y propuesta de medidas}

La fase final cierra el proyecto y propone acciones futuras.

\subsection{8.1 Evaluación final del rendimiento del proyecto}

Se evalúa el grado de cumplimiento de los objetivos y la calidad del proceso seguido.

\subsection{8.2 Elaboración del informe final}

Se entrega un informe final que resume resultados, conclusiones y recomendaciones.
