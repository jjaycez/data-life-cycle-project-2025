\chapter{Fuentes de datos}\label{ch:fuentes-de-datos}
% TODO: decidir si es un título adecuado, he usado uno genérico

En este capítulo se describen las fuentes de datos utilizadas en el análisis del impacto de la contaminación
atmosférica en los casos de asma en California.

Se detallan las características de cada conjunto de datos, su origen, y cómo se integran en el estudio.
En este estudio se han empleado principalmente dos fuentes de datos:
\begin{itemize}
    \item Datos de calidad del aire obtenidos de la iniciativa~\openaq.
    \item
        Datos de ingresos hospitalarios relacionados con asma proporcionados por \datagov, la plataforma de datos
        abiertos del gobierno de Estados Unidos.
\end{itemize}

Para la extracción y filtrado de los datos, se han realizado una serie de tareas que se describirán en detalle en las
secciones siguientes.

Todo este proceso ha sido recogido en un cuaderno de Jupyter, que se puede localizar en el repositorio del proyecto,
en la ruta \texttt{code/ETL.ipynb}.

\bigskip

A lo largo de las secciones, se facilitarán algunas capturas de pantalla de fragmentos del cuaderno para ilustrar
ciertas partes del proceso.

\section{Datos de calidad del aire}\label{sec:datos-de-calidad-del-aire}

Los datos de calidad del aire se han obtenido de la iniciativa~\openaq, que proporciona acceso abierto a
mediciones de contaminantes atmosféricos a nivel global.

\smallskip

Entre otros contaminantes, las entidades colaboradoras exponen datos sobre niveles de dióxido de nitrógeno ($NO_{2}$),
ozono ($O_{3}$) y material particulado fino ($PM_{2.5}$), que son relevantes para el estudio del asma.

\smallskip

Además de ofrecer un visor web, \openaq cuenta con una API que permite la descarga programática de datos históricos
y en tiempo real.
En la documentación oficial de la API se detallan los \textit{endpoints} disponibles, los parámetros de consulta y los
formatos de respuesta.

\smallskip

En concreto, es de particular relevancia el hecho de que la API tiene un límite de peticiones para distintas franjas
temporales.
En el caso de California, ha sido de relevancia considerar los siguientes límites, cuyas razones revelaremos más adelante:

\begin{itemize}
    \item 60 peticiones por minuto.
    \item 2000 peticiones por hora.
\end{itemize}

Estas restricciones han sido el principal cuello de botella a la hora de descargar grandes volúmenes de datos
históricos.

\subsection{Geografía de California}\label{subsec:geografia-california}

Los datos expuestos por el gobierno de Estados Unidos están desglosados por condados, que son las subdivisiones
administrativas de primer nivel dentro de los estados.

\bigskip

Esto es de relevancia, puesto que las estaciones de medición de calidad del aire exponen su ubicación geográfica en
coordenadas de latitud y longitud (EPSG:4326), por lo que es necesario asignar cada estación a su condado
correspondiente.

Para este cometido, se han realizado dos tareas principales:

\begin{itemize}
    \item
        \textbf{Obtener los límites geográficos de los condados de California.}
        
        Para ello, se han empleado los datos de \textit{shapefiles} proporcionados por el
        \textit{U.S. Census Bureau}\footnote{\url{https://www2.census.gov/geo/tiger/TIGER2025/COUNTY/tl_2025_us_county.zip}}.
        % TODO: buscar cómo integrarlo en el sistema de referencias bibliográficas
        Estos archivos contienen la geometría de los condados, que se ha utilizado para determinar si una estación
        de medición se encuentra dentro de un condado específico.
        
    \item
        \textbf{Filtrar las estaciones de medición mediante un bbox}

        (\textit{bounding box}, i.e., un rectángulo delimitador) que englobe toda California, puesto que la API de
        \openaq permite filtrar por este criterio.
\end{itemize}

\begin{comment}
    - Describir el formato de los shapefiles
    - Indicar que California se corresponde con el estado con código FIPS 06
    - Indicar que se ha empleado `shapely` para determinar si un punto está dentro de un polígono
    - Mostrar una distribución de estaciones por condado
\end{comment}

A continuación se muestra el código empleado para descargar los shapefiles de los condados:

\begin{minted}{python}

import requests
from pathlib import Path
# download county shapefiles from US Census Bureau
shapes_url = 'https://www2.census.gov/geo/tiger/TIGER2025/COUNTY/tl_2025_us_county.zip'
geo_dir = Path("data/geographical")
geo_dir.mkdir(parents=True, exist_ok=True)
shapes_path = geo_dir / "tl_2025_us_county.zip"
if not shapes_path.exists():
    print(f"Downloading county shapefiles from {shapes_url}...")
    r = requests.get(shapes_url)
    with open(shapes_path, 'wb') as f:
        f.write(r.content)
    print(f"Downloaded to {shapes_path}")
else:
    print(f"Shapefiles already exist at {shapes_path}")
   
\end{minted}

El archivo comprimido descargado se encuentra en la ruta \path{data/geographical/tl_2025_us_county.zip},
cuya
versión descomprimida se halla en la sub-carpeta \path{california_counties}.

\subsection{Descarga de datos históricos}\label{subsec:descarga-datos-historicos}

\begin{comment}
    - Explicar la cabecera `X-Rate-Limit-Remaining` y cómo se ha empleado para gestionar las peticiones
    - Explicar el tiempo estimado para descargar los datos de toda California
    - Explicar que el código se tuvo que modificar varias veces, pero no se perdió el progreso gracias a los archivos
      intermedios, por eso la barra de progreso muestra un tiempo menor del real
    - Explicar que son 5333 archivos JSON, por lo que se ha optado comprimirlos en un ZIP para facilitar su manipulación
    - Mostrar como ejemplo la estructura de una respuesta en formato JSON
\end{comment}

\subsection{Procesamiento de los datos descargados}\label{subsec:procesamiento-datos-descargados}

\begin{comment}
    - Explicar que se ha creado una base de datos SQLite para facilitar el análisis posterior
    - Mostrar un esquema de la base de datos
    - REMINDER: los datos no han sido curados, solo se han insertado tal cual
    - Explicar, si procede, las decisiones de diseño tomadas (tipos de datos, índices, etc.)
\end{comment}

\section{Datos de prevalencia de asma en California}\label{sec:datos-asma}

\begin{comment}
    - Explicar la estructura de los datos obtenidos de data.gov
    - TODO: procesar los datos en la base de datos SQLite
    - Mostrar un esquema actualizado de la base de datos
    - Explicar las decisiones de diseño tomadas (tipos de datos, índices, etc.)
\end{comment}