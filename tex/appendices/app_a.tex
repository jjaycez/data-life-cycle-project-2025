\chapter{Código para la fase ETL}\label{ch:apendice-a:-codigo-para-la-fase-etl}
En este apéndice se presenta el código utilizado para la fase de Extracción, Transformación y Carga (ETL)
de los datos en nuestro proyecto.

Las variables no definidas en los fragmentos de código se pueden consultar en la siguiente lista:
\begin{itemize}
    \item \texttt{API\_KEY}: Clave de API para acceder a la API de \openaq.
    \item \texttt{selected\_stations}: DataFrame de pandas que contiene las estaciones seleccionadas para la descarga de datos históricos.
\end{itemize}

\begin{listing}
    \begin{minted}{python}
# define bounding box for California
x_range = (-125, -113)
y_range = (32, 42.5)

import requests

BASE = "https://api.openaq.org/v3"
headers = {"X-API-Key": API_KEY}

all_results = []
page = 1
limit = 1000   # OpenAQ v3 maximum

while True:
    r = requests.get(
        f"{BASE}/locations",
        headers=headers,
        params={
            "bbox": f"{x_range[0]},{y_range[0]},{x_range[1]},{y_range[1]}",
            "limit": limit,
            "page": page,
            "country": "US",   # optional but helps validation
        },
        timeout=60,
    )
    r.raise_for_status()
    data = r.json()

    results = data.get("results", [])
    if not results:
        break

    all_results.extend(results)
    print(f"page {page}: {len(results)}")

    page += 1

print("TOTAL returned:", len(all_results))
print("first name:", all_results[0].get("name"))
    \end{minted}
    \caption{Petición a la API de \openaq dentro de un \bboxt}
    \label{lst:openaq_extraction}
\end{listing}
