\chapter{Código para la fase ETL}\label{ch:apendice-a:-codigo-para-la-fase-etl}
En este apéndice se presenta el código utilizado para la fase de Extracción, Transformación y Carga (ETL)
de los datos en nuestro proyecto.

Las variables no definidas en los fragmentos de código se pueden consultar en la siguiente lista:
\begin{itemize}
    \item \texttt{API\_KEY}: Clave de API para acceder a la API de \openaq.
    \item \texttt{selected\_stations}: DataFrame de pandas que contiene las estaciones seleccionadas para la descarga de datos históricos.
\end{itemize}

\newpage
% --- Extracción de datos de OpenAQ ---
\section{Extracción de datos de OpenAQ}\label{sec:extraccion-de-datos-de-openaq}

\begin{minted}{python}
# define bounding box for California
x_range = (-125, -113)
y_range = (32, 42.5)
import requests
BASE = "https://api.openaq.org/v3"
headers = {"X-API-Key": API_KEY}
all_results = []
page = 1
limit = 1000   # OpenAQ v3 maximum
while True:
    r = requests.get(
        f"{BASE}/locations",
        headers=headers,
        params={
            "bbox": f"{x_range[0]},{y_range[0]},{x_range[1]},{y_range[1]}",
            "limit": limit,
            "page": page,
            "country": "US",   # optional but helps validation
        },
        timeout=60,
    )
    r.raise_for_status()
    data = r.json()
    
    results = data.get("results", [])
    if not results:
        break
    
    all_results.extend(results)
    print(f"page {page}: {len(results)}")
    
    page += 1

print("TOTAL returned:", len(all_results))
print("first name:", all_results[0].get("name"))
\end{minted}
\newpage
% --- Esquema de primera fuente de datos ---
\section{Esquema de primera fuente de datos}\label{sec:schema-de-primera-fuente-de-datos}
\begin{minted}{sql}
CREATE TABLE IF NOT EXISTS units(
    unit_id INTEGER PRIMARY KEY,
    unit_name VARCHAR(100) NOT NULL,
    description TEXT
);

CREATE TABLE IF NOT EXISTS pollutants(
    pollutant_id INTEGER PRIMARY KEY,
    pollutant_name VARCHAR(100) NOT NULL
);

CREATE TABLE IF NOT EXISTS stations(
    station_id INTEGER PRIMARY KEY,
    county_id INTEGER,
    latitude DECIMAL(9,6),
    longitude DECIMAL(9,6),
    FOREIGN KEY (county_id) REFERENCES counties(county_id)
);

CREATE TABLE IF NOT EXISTS counties(
    county_id INTEGER PRIMARY KEY,
    county_name VARCHAR(100) NOT NULL,
    geometry TEXT
);

CREATE TABLE IF NOT EXISTS sensors(
    sensor_id INTEGER PRIMARY KEY,
    station_id INTEGER,
    pollutant_id INTEGER,
    unit_id INTEGER,
    FOREIGN KEY (station_id) REFERENCES stations(station_id),
    FOREIGN KEY (pollutant_id) REFERENCES pollutants(pollutant_id),
    FOREIGN KEY (unit_id) REFERENCES units(unit_id)
);

CREATE TABLE IF NOT EXISTS measurements(
    measurement_id INTEGER PRIMARY KEY,
    sensor_id INTEGER,
    measurement_value DECIMAL(10,4) NOT NULL,
    start_time DATETIME NOT NULL,
    end_time DATETIME NOT NULL,
    FOREIGN KEY (sensor_id) REFERENCES sensors(sensor_id),
    UNIQUE(sensor_id, start_time, end_time)
);
\end{minted}
\newpage
% --- Schema de segunda fuente de datos ---
\section{Esquema de segunda fuente de datos}\label{sec:schema-de-segunda-fuente-de-datos}
\begin{minted}{sql}
CREATE TABLE IF NOT EXISTS asthma_prevalence (
    id INTEGER PRIMARY KEY,
    county_id INTEGER,
    year_from INTEGER,
    year_to INTEGER,
    demographic_group_id INTEGER,
    current_prevalence FLOAT,
    ci_95_lower FLOAT,
    ci_95_upper FLOAT,
    comment TEXT,
    FOREIGN KEY (county_id) REFERENCES counties(county_id),
    FOREIGN KEY (demographic_group_id) REFERENCES demographic_group(id)
);

CREATE TABLE IF NOT EXISTS grouped_counties (
    -- Though it's a 1:N relationship, we create a separate table to allow for easier querying and future expansion
    county_id INTEGER,
    asthma_prevalence_id INTEGER,
    FOREIGN KEY (county_id) REFERENCES counties(county_id),
    FOREIGN KEY (asthma_prevalence_id) REFERENCES asthma_prevalence(id),
    PRIMARY KEY (county_id, asthma_prevalence_id)
);

CREATE TABLE IF NOT EXISTS demographic_group (
    id INTEGER PRIMARY KEY,
    strata TEXT,
    age_group TEXT, -- as is from the data
    age_min INTEGER,
    age_max INTEGER
);
\end{minted}